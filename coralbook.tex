\documentclass[]{book}
\usepackage{lmodern}
\usepackage{amssymb,amsmath}
\usepackage{ifxetex,ifluatex}
\usepackage{fixltx2e} % provides \textsubscript
\ifnum 0\ifxetex 1\fi\ifluatex 1\fi=0 % if pdftex
  \usepackage[T1]{fontenc}
  \usepackage[utf8]{inputenc}
\else % if luatex or xelatex
  \ifxetex
    \usepackage{mathspec}
  \else
    \usepackage{fontspec}
  \fi
  \defaultfontfeatures{Ligatures=TeX,Scale=MatchLowercase}
\fi
% use upquote if available, for straight quotes in verbatim environments
\IfFileExists{upquote.sty}{\usepackage{upquote}}{}
% use microtype if available
\IfFileExists{microtype.sty}{%
\usepackage{microtype}
\UseMicrotypeSet[protrusion]{basicmath} % disable protrusion for tt fonts
}{}
\usepackage[margin=1in]{geometry}
\usepackage{hyperref}
\hypersetup{unicode=true,
            pdftitle={A Minimal Book Example},
            pdfauthor={The coral team, O'Dea lab},
            pdfborder={0 0 0},
            breaklinks=true}
\urlstyle{same}  % don't use monospace font for urls
\usepackage{natbib}
\bibliographystyle{apalike}
\usepackage{color}
\usepackage{fancyvrb}
\newcommand{\VerbBar}{|}
\newcommand{\VERB}{\Verb[commandchars=\\\{\}]}
\DefineVerbatimEnvironment{Highlighting}{Verbatim}{commandchars=\\\{\}}
% Add ',fontsize=\small' for more characters per line
\usepackage{framed}
\definecolor{shadecolor}{RGB}{248,248,248}
\newenvironment{Shaded}{\begin{snugshade}}{\end{snugshade}}
\newcommand{\KeywordTok}[1]{\textcolor[rgb]{0.13,0.29,0.53}{\textbf{{#1}}}}
\newcommand{\DataTypeTok}[1]{\textcolor[rgb]{0.13,0.29,0.53}{{#1}}}
\newcommand{\DecValTok}[1]{\textcolor[rgb]{0.00,0.00,0.81}{{#1}}}
\newcommand{\BaseNTok}[1]{\textcolor[rgb]{0.00,0.00,0.81}{{#1}}}
\newcommand{\FloatTok}[1]{\textcolor[rgb]{0.00,0.00,0.81}{{#1}}}
\newcommand{\ConstantTok}[1]{\textcolor[rgb]{0.00,0.00,0.00}{{#1}}}
\newcommand{\CharTok}[1]{\textcolor[rgb]{0.31,0.60,0.02}{{#1}}}
\newcommand{\SpecialCharTok}[1]{\textcolor[rgb]{0.00,0.00,0.00}{{#1}}}
\newcommand{\StringTok}[1]{\textcolor[rgb]{0.31,0.60,0.02}{{#1}}}
\newcommand{\VerbatimStringTok}[1]{\textcolor[rgb]{0.31,0.60,0.02}{{#1}}}
\newcommand{\SpecialStringTok}[1]{\textcolor[rgb]{0.31,0.60,0.02}{{#1}}}
\newcommand{\ImportTok}[1]{{#1}}
\newcommand{\CommentTok}[1]{\textcolor[rgb]{0.56,0.35,0.01}{\textit{{#1}}}}
\newcommand{\DocumentationTok}[1]{\textcolor[rgb]{0.56,0.35,0.01}{\textbf{\textit{{#1}}}}}
\newcommand{\AnnotationTok}[1]{\textcolor[rgb]{0.56,0.35,0.01}{\textbf{\textit{{#1}}}}}
\newcommand{\CommentVarTok}[1]{\textcolor[rgb]{0.56,0.35,0.01}{\textbf{\textit{{#1}}}}}
\newcommand{\OtherTok}[1]{\textcolor[rgb]{0.56,0.35,0.01}{{#1}}}
\newcommand{\FunctionTok}[1]{\textcolor[rgb]{0.00,0.00,0.00}{{#1}}}
\newcommand{\VariableTok}[1]{\textcolor[rgb]{0.00,0.00,0.00}{{#1}}}
\newcommand{\ControlFlowTok}[1]{\textcolor[rgb]{0.13,0.29,0.53}{\textbf{{#1}}}}
\newcommand{\OperatorTok}[1]{\textcolor[rgb]{0.81,0.36,0.00}{\textbf{{#1}}}}
\newcommand{\BuiltInTok}[1]{{#1}}
\newcommand{\ExtensionTok}[1]{{#1}}
\newcommand{\PreprocessorTok}[1]{\textcolor[rgb]{0.56,0.35,0.01}{\textit{{#1}}}}
\newcommand{\AttributeTok}[1]{\textcolor[rgb]{0.77,0.63,0.00}{{#1}}}
\newcommand{\RegionMarkerTok}[1]{{#1}}
\newcommand{\InformationTok}[1]{\textcolor[rgb]{0.56,0.35,0.01}{\textbf{\textit{{#1}}}}}
\newcommand{\WarningTok}[1]{\textcolor[rgb]{0.56,0.35,0.01}{\textbf{\textit{{#1}}}}}
\newcommand{\AlertTok}[1]{\textcolor[rgb]{0.94,0.16,0.16}{{#1}}}
\newcommand{\ErrorTok}[1]{\textcolor[rgb]{0.64,0.00,0.00}{\textbf{{#1}}}}
\newcommand{\NormalTok}[1]{{#1}}
\usepackage{longtable,booktabs}
\usepackage{graphicx,grffile}
\makeatletter
\def\maxwidth{\ifdim\Gin@nat@width>\linewidth\linewidth\else\Gin@nat@width\fi}
\def\maxheight{\ifdim\Gin@nat@height>\textheight\textheight\else\Gin@nat@height\fi}
\makeatother
% Scale images if necessary, so that they will not overflow the page
% margins by default, and it is still possible to overwrite the defaults
% using explicit options in \includegraphics[width, height, ...]{}
\setkeys{Gin}{width=\maxwidth,height=\maxheight,keepaspectratio}
\IfFileExists{parskip.sty}{%
\usepackage{parskip}
}{% else
\setlength{\parindent}{0pt}
\setlength{\parskip}{6pt plus 2pt minus 1pt}
}
\setlength{\emergencystretch}{3em}  % prevent overfull lines
\providecommand{\tightlist}{%
  \setlength{\itemsep}{0pt}\setlength{\parskip}{0pt}}
\setcounter{secnumdepth}{5}
% Redefines (sub)paragraphs to behave more like sections
\ifx\paragraph\undefined\else
\let\oldparagraph\paragraph
\renewcommand{\paragraph}[1]{\oldparagraph{#1}\mbox{}}
\fi
\ifx\subparagraph\undefined\else
\let\oldsubparagraph\subparagraph
\renewcommand{\subparagraph}[1]{\oldsubparagraph{#1}\mbox{}}
\fi

%%% Use protect on footnotes to avoid problems with footnotes in titles
\let\rmarkdownfootnote\footnote%
\def\footnote{\protect\rmarkdownfootnote}

%%% Change title format to be more compact
\usepackage{titling}

% Create subtitle command for use in maketitle
\newcommand{\subtitle}[1]{
  \posttitle{
    \begin{center}\large#1\end{center}
    }
}

\setlength{\droptitle}{-2em}
  \title{A Minimal Book Example}
  \pretitle{\vspace{\droptitle}\centering\huge}
  \posttitle{\par}
  \author{The coral team, O'Dea lab}
  \preauthor{\centering\large\emph}
  \postauthor{\par}
  \predate{\centering\large\emph}
  \postdate{\par}
  \date{2016-12-30}

\usepackage{booktabs}

\begin{document}
\maketitle

{
\setcounter{tocdepth}{1}
\tableofcontents
}
\chapter{Prerequisites}\label{prerequisites}

This is a \emph{sample} book written in \textbf{Markdown}. You can use
anything that Pandoc's Markdown supports, e.g., a math equation
\(a^2 + b^2 = c^2\).

For now, you have to install the development versions of
\textbf{bookdown} from Github:

\begin{Shaded}
\begin{Highlighting}[]
\NormalTok{devtools::}\KeywordTok{install_github}\NormalTok{(}\StringTok{"rstudio/bookdown"}\NormalTok{)}
\end{Highlighting}
\end{Shaded}

Remember each Rmd file contains one and only one chapter, and a chapter
is defined by the first-level heading \texttt{\#}.

To compile this example to PDF, you need to install XeLaTeX.

\chapter{Introduction}\label{intro}

You can label chapter and section titles using \texttt{\{\#label\}}
after them, e.g., we can reference Chapter \ref{intro}. If you do not
manually label them, there will be automatic labels anyway, e.g.,
Chapter \ref{methods}.

Figures and tables with captions will be placed in \texttt{figure} and
\texttt{table} environments, respectively.

\begin{Shaded}
\begin{Highlighting}[]
\KeywordTok{par}\NormalTok{(}\DataTypeTok{mar =} \KeywordTok{c}\NormalTok{(}\DecValTok{4}\NormalTok{, }\DecValTok{4}\NormalTok{, .}\DecValTok{1}\NormalTok{, .}\DecValTok{1}\NormalTok{))}
\KeywordTok{plot}\NormalTok{(pressure, }\DataTypeTok{type =} \StringTok{'b'}\NormalTok{, }\DataTypeTok{pch =} \DecValTok{19}\NormalTok{)}
\end{Highlighting}
\end{Shaded}

\begin{figure}

{\centering \includegraphics[width=0.8\linewidth]{coralbook_files/figure-latex/nice-fig-1} 

}

\caption{Here is a nice figure!}\label{fig:nice-fig}
\end{figure}

Reference a figure by its code chunk label with the \texttt{fig:}
prefix, e.g., see Figure \ref{fig:nice-fig}. Similarly, you can
reference tables generated from \texttt{knitr::kable()}, e.g., see Table
\ref{tab:nice-tab}.

\begin{Shaded}
\begin{Highlighting}[]
\NormalTok{knitr::}\KeywordTok{kable}\NormalTok{(}
  \KeywordTok{head}\NormalTok{(iris, }\DecValTok{20}\NormalTok{), }\DataTypeTok{caption =} \StringTok{'Here is a nice table!'}\NormalTok{,}
  \DataTypeTok{booktabs =} \OtherTok{TRUE}
\NormalTok{)}
\end{Highlighting}
\end{Shaded}

\begin{table}

\caption{\label{tab:nice-tab}Here is a nice table!}
\centering
\begin{tabular}[t]{rrrrl}
\toprule
Sepal.Length & Sepal.Width & Petal.Length & Petal.Width & Species\\
\midrule
5.1 & 3.5 & 1.4 & 0.2 & setosa\\
4.9 & 3.0 & 1.4 & 0.2 & setosa\\
4.7 & 3.2 & 1.3 & 0.2 & setosa\\
4.6 & 3.1 & 1.5 & 0.2 & setosa\\
5.0 & 3.6 & 1.4 & 0.2 & setosa\\
\addlinespace
5.4 & 3.9 & 1.7 & 0.4 & setosa\\
4.6 & 3.4 & 1.4 & 0.3 & setosa\\
5.0 & 3.4 & 1.5 & 0.2 & setosa\\
4.4 & 2.9 & 1.4 & 0.2 & setosa\\
4.9 & 3.1 & 1.5 & 0.1 & setosa\\
\addlinespace
5.4 & 3.7 & 1.5 & 0.2 & setosa\\
4.8 & 3.4 & 1.6 & 0.2 & setosa\\
4.8 & 3.0 & 1.4 & 0.1 & setosa\\
4.3 & 3.0 & 1.1 & 0.1 & setosa\\
5.8 & 4.0 & 1.2 & 0.2 & setosa\\
\addlinespace
5.7 & 4.4 & 1.5 & 0.4 & setosa\\
5.4 & 3.9 & 1.3 & 0.4 & setosa\\
5.1 & 3.5 & 1.4 & 0.3 & setosa\\
5.7 & 3.8 & 1.7 & 0.3 & setosa\\
5.1 & 3.8 & 1.5 & 0.3 & setosa\\
\bottomrule
\end{tabular}
\end{table}

You can write citations, too. For example, we are using the
\textbf{bookdown} package \citep{R-bookdown} in this sample book, which
was built on top of R Markdown and \textbf{knitr} \citep{xie2015}.

\section{Setting}\label{setting}

`\texttt{re-written\ from\ @Guzman\_etal\_2005,\ who\ cited\ nobody:}The
province of Bocas del Toro is located in western Panama (8°30-9°40 N,
82°56-81°18 W, xxxfig. map). It borders to the west with Costa Rica, to
the south and east with the provinces of Chiriquí and Veraguas,
respectively, and to the north with the Caribbean sea.

\texttt{Generally\ re-written\ from\ @Cramer\_2013,\ who\ cited\ nobody:}
The archipelago of Bocas del Toro has seven medium-sized islands
(\texttt{re-writen\ from\ @Schloder\_etal\_2013\ who\ cited\ nobody}),
estuaries and beaches.

\section{Geology}\label{geology}

\texttt{@Guzman\_etal\_2005:}About 20 Ma ago--before the isthmus of
Panama formed--the Bocas del Toro Basin was a deep tropical marine bed
(Coates and Jackson 1998, cited in \citet{Guzman_etal_2005}). Volcanoes
were active between 16-10 Ma and many parts of the archipelago have
igneous and sedimentary rocks. When volcanic activity ceased, marine
transgressions deposited a sedimentary sequence near the coast, adequate
for (a) diverse and abundant marine fauna to settle, (b) new species of
reef coral and benthic foraminifers to originate, and (c) reefs to form
(Coates and Jackson 1998, Collins et al. 1996, cited in
\citet{Guzman_etal_2005}).

Sea level changed substantially over the past 9500 years and strongly
shaped the Bocas del Toro archipelago in important ways, and affected
the bathymetry, topography and separation between islands and the
mainland.

The current shape of the archipelago settled when sea level stabilized
around 6000 years ago, in the mid-Holocene (Summers et al. 1997, cited
in \citet{Guzman_etal_2005}). Evidenced of this was found in Isla Colon,
where multiple corals from a fossil reef dated within the mid-Holocene
consistently \citep{Fredston_etal_2013}.

Today, reefs range from patch to fringing and from environments exposed
to off-shore swell--typically, with sandy beaches or reef flats--to
environments protected into lagoons--typically, with fringing mangroves
and extensive seagrass beds.

\section{Geography}\label{geography}

\texttt{re-written\ from\ @Cramer\_2013,\ who\ cited\ nobody:}Because
mountain ranges divide each side the isthmus of Panama, the drainage
basins and microclimates are distinct towards the Pacific and Caribbean.
\texttt{re-writen\ from\ @Guzman\_etal\_2005:}Near the Caribbean
littoral and in Bocas del Toro, a mountain chain (50-400 m high) runs
parallel to the coast and closer than 3.5 km from Chiriquí and Bahía
Almirante lagoons (xxx IGNTG 1988 cited in \citet{Guzman_etal_2005}).

Chiriquí and Bahía Almirante are the two semi-enclosed lagoons in which
the archipelago of Bocas del Toro is divided. They are connected through
mangrove channels and they both open to the ocean (Carruthers et al.
2005, D'Croz et al. 2005, cited in \citet{Schloder_etal_2013}).

\texttt{re-writen\ from\ @Guzman\_etal\_2005:}Both lagoons receive
several rivers. Because the distance between the mountains and coast is
short, river flow after the rains increases immediately. This forms a
lens of superficial water that is approximately 0.5 m thick and rich in
suspended organic material. This affects both lagoons but more strongly
to because the large Cricamola river feeds directly into its
lagoon\texttt{(re-writen\ from\ @Guzman\_etal\_2005\ and\ @Cramer\_2013)}.
Yet, Bahía Almirante receives creeks that drain the floodplain of the
Changuinola River (northwest of the bay) through extensive banana
plantations and input sediments and pollutants (Guzmán and Jimenez 1992,
Guzmán and Garcia 2002, Guzmán 2003, cited in \citet{Cramer_2013}).

\section{Climate}\label{climate}

The climate is typical of the wet Caribbean. Precipitation in Bocas del
Toro is 330 cm annually (Cubit et al. 1989, Kaufmann and Thompson 2005,
cited in \citet{Cramer_2013}).

The dry and rainy seasons are not clearly defined. Generally, rainfall
is lower in March and between September-October and higher in July and
December (Kaufman and Thompson, cited in \citet{Guzman_etal_2005})

\section{Vegetation}\label{vegetation}

The vegetation is mainly banana plantations, clear-felled cattle lands
and rainforest on the mainland and lowland wet tropical-forest on the
islands (xxx Carruthers et al. 2005, xxxCramer et al. 2012, xxx Cramer
2013 cited in \citet{Schloder_etal_2013}).

\section{Oceanography}\label{oceanography}

\texttt{re-writen\ from\ @Guzman\_etal\_2005:}The continental shelf in
Bocas del Toro is narrow and the maximum depth along the coast ranges
20-50 m. Surf and tides are more important outside than inside the
archipelago. Tides may be mixed, semidiurnal or diurnal. Their amplitude
ranges below 0.5 m and are higher than usual during the dry season
because of the Northeast winds (xxx Glynn 1972 cited in
\citet{Guzman_etal_2005}).

\texttt{re-writen\ from\ @Guzman\_etal\_2005:}The main current that
influences the coast of Panama flows from West to East and, because it
comes closer to its coast, it may affect Bocas del Toro more strongly
from June-August(DMA 1988; Greb et al. 1996, cited in
\citet{Guzman_etal_2005}).

Winds that more strongly influence the archipelago blow from the North
and Northeast but their effect on tides and surf is reduced by the
northern islands and reefs (DMA 1988, cited in
\citet{Guzman_etal_2005}).

\section{Caribbean reefs}\label{caribbean-reefs}

Since the 1970s, coral reefs in the Caribbean changed massively: the
cover of live corals decreased and that of macroalgae increased in
unprecedented manner \citep{Jackson_etal_2014}(Aronson and Precht 1997,
Greenstein et al. 1998, cited in \citet{Fredston_etal_2013}). This is
anomalous (Silvia Earle, 1972) and indicates that reefs have severely
declined \citep{Jackson_etal_2014}. This degradation was ultimately
caused by human impacts, strongly linked to the rocketing growth of
human population, tourism and shipping, via overexplotation of natural
resources and spread of disease \citep{Jackson_etal_2014}. The most
likely mechanisms and most evident proximate causes unfolded in the
following sequence.

\texttt{re-writen\ from\ @Jackson\_etal\_2014,\ excecutive\ summary,\ who\ cited\ nobody\ (need\ to\ go\ to\ the\ extended\ version\ for\ references:}Between
mid 1970s and early 1980s, enormous shipping through the Panama Canal
facilitated alien species, including disease, to invade the Caribbean
sea. White band disease infected and massively killed \emph{Acropora}
corals. An unidentified disease also infected and massively killed
\emph{Diadema} sea urchins. \emph{Diadema} and parrot fish are both
hervivores and key in regulating the balance between macroalgae and
corals in reefs. Because humans had overexploited parrot fish, when
\emph{Diadema} died-off, macroalgae bloomed and reefs--originally
dominated by corals--became dominated by macroalge. This phase-shift
peaked by mid 1990s and continues today. Recently, greater overfishing,
pollution, numbers of tourists and of extreeme warming events have
worsen coral reefs.

\section{Human ocupation and impact}\label{human-ocupation-and-impact}

Humans have potentially affected land and sea in Bocas del Toro ever
since sea level stabilized around its current position and coral reefs
developed but human impact here has become dramatic only recently
\citep{Cramer_2013}.

Humans have cleared the land and fished marine coasts in Central America
since around 10,000 years ago \citep{Cramer_2013}. By then, they had
already occupied the Pacific coast of Panama \citep{Fredston_etal_2013}.
In Bocas del Toro, however, the earliest hunters and gatherers were
recorded around 5000 years ago (Ranere and Cooke 1991, Cooke 2005, cited
in \citet{Fredston_etal_2013}) and the earliest human settlements and
fishing activities, around 1500 years ago (Wake et al. 2004, 2013, cited
in \citet{Fredston_etal_2013}).

When the Spanish arrived to America, approximately 500 years ago, human
population in Bocas del Toro had depleted the largest animals from
mangroves, seagrass meadows, and coral reefs (Wing and Wing 2001,
Pandolfi et al. 2003, cited in \citet{Cramer_2013}).

Then, the most catastrophic mortality that resulted from European
contact was that of indigenous peoples. Population recovered pre-contact
levels in the 1800s \citep{Cramer_2013} but it was distributed sparsely
until 1900s--because the orography and climate of Bocas del Toro was
challenging for human settlement and agriculture--(\citet{Cramer_2013}).
Soon after, human activities caused rapid environmental change and had
the greatest impact on ecosystems in Bocas del Toro.

\texttt{re-writen\ from\ @Seemann\_etal\_2014:}Since 1915, mainland and
islands in Bocas del Toro have been deforested to grow and export
bananas. Channels have been dredged and ships have been departing from
Almirante port, crossing Bahia Almirante and exiting into the Caribbean
sea through Bocas del Drago (xxx Greb et al. 1996, cited in
\citet{Seemann_etal_2014}). Cumulatively, deforestation, dredging and
shipping has caused erosion, sediments, nutrients and pollution to
increase (xxx Berry et al. 2013, xxx Burke et al. 2004, , cited in
\citet{Seemann_etal_2014}). Since 1993, population and tourism have
dramatically increased (xxx Guerrón-Montero 2005, cited in
\citet{Seemann_etal_2014}), coasts extensively developed, marine
resources overfished and destructive fishing methods used. Consequently,
ecosystems have substantially changed over the past few decades, (xxx
Saric 2005, cited in \citet{Seemann_etal_2014}).

\section{Ecology}\label{ecology}

A detailed description of the ecological structure of coral reefs in
Bocas del Toro is available elsewhere (including Laguna de Chiriqui:
\citet{Guzman_etal_1998_a}, Islas Bastimentos, Solarte, Carenero and
Colón: \citet{Guzman_etal_1998_b}, islas Pastores, Cristóbal, Popa and
Cayo Agua: \citet{Guzman_etal_1999}, and Península Valiente:
\citet{Guzman_etal_2001}).

\subsection{Bahia Almirante}\label{bahia-almirante}

Bahía Almirante hosts 53\% (33) of all the species of reef coral found
in Panama \citep[\citet{Guzman_etal_1998_a}]{Guzman_etal_1999}. In this
bay, reefs are more extensively developed than in Chiriqui Lagoon--which
receives river runoff more directly (Guzmán and Guevara 1998a, D'Croz et
al. 2005, Guzmán et al. 2005, \citet{Cramer_2013}). However, vertical
development is limited to a maximum depth of 23 m, below that light is
insufficient for corals to grow (Guzman\_etal\_1998\_a). Around the
beginning of the 21st century, coral cover ranged 20-50\% and averaged
35\% \citep{Guzman_etal_1998_a}. Species distributed vertically
similarly across the bay: \emph{Porites furcata} (90\% cover) dominated
down to 2 m depth and \emph{Agaricia tenuifolia} (40\% cover) from 2-6
m. The depth range from 6-15 m was composed by \emph{A. tenuifolia},
\emph{Madracis mirabilis}, and \emph{Siderastrea siderea} (15\%, 7\% and
5\%, respectively). \emph{S. siderea} and coralline algae dominated
deeper waters. Sponges where the second most important sessile organism
and were found in all reefs.

\subsection{Isla Colon}\label{isla-colon}

Facing Bahia Almirante and along the leeward coast of Isla Colon, reefs
form an almost continuous barrier \citep{Guzman_etal_1998_a}. Around the
center of the island, corals and algae covered 26.9\% and 21.4\% of the
reefs in mid 2000s, while richness of reef coral species was 17
\citep{Guzman_etal_2005}.

\subsection{Recent status}\label{recent-status}

More recently, between 2010-2011, cover and richness of reef corals in
Bahia Almirante was \textless{}10\% and 7 species, respectively
\citep{Seemann_etal_2014}. \emph{Porites furcata} was dominant reef
coral and adult and carnivorous fish species were absent
\citep{Seemann_etal_2014}.

\subsection{Main environmental factors affecting the
ecology}\label{main-environmental-factors-affecting-the-ecology}

The ecology of coral reefs in Bahia Almirante, in particular, is
strongly influenced by the Changuinola river. Its plumes enter the bay
through an inlet at Boca del Drago \citep{Seemann_etal_2014}. (This,
however, does not explain why Punta Caracol is healthier than Casa
Blanca and has \emph{Acropora cervicornis} which is assumed to be highly
sensitive to sedimentation.)

The environmental pressure of riverine waters can accumulate with other
factors and impact reefs ecology. For example, coral bleaching in
September 2010 killed corals worldwide. Coral mortality and degradation
of reef ecosystems was particularly extensive in Bocas del Toro (NOAA
2010, cited in \citet{Seemann_etal_2014}). Here, not only sea
temperatures increased but also salinity lowered because seasonal rain
fall was high and oxygen depleted because water exchange was limited
(Kaufmann and Thompson 2005, cited in \citet{Seemann_etal_2014}).

\section{Shifts in ecology}\label{shifts-in-ecology}

Until recently, \emph{Acropora cervicornis} (Lamarck, 1816) and
\emph{Acropora palmata} (Lamarck, 1816) had dominated fringing or
lagoonal reefs in Bocas del Toro ever since around 2 Ma. Importantly,
they dominated since the mid-Holocene (\textasciitilde{}8000-4000 years
ago, xxxref), when sea level stabilized around its current position and
reefs comparable to modern reefs developed (Aronson et al. 2004, Klaus
et al. 2011, Cramer et al. 2012, Cramer 2013, Fredston-Hermann et al.
2013, cited in \citet{Schloder_etal_2013}).

For example, Lennond reef--a mid-Holocene reef from the leeward side of
Isla Colon, in Bahia Almirante--was dominated by \emph{A. cervicornis},
\emph{Montastraea ``annularis''} (Ellis and Solander, 1786), and
\emph{Porites furcata} \citep{Fredston_etal_2013}. These assembly of
species suggests that Lennond reef was 0-15 m deep (Mesolella 1967,
Aronson et al. 2004, cited in \citet{Fredston_etal_2013}). This is
considering that some species inhabit shallower waters than typical if
the environment is protected. For example, \emph{A. cervicornis} occurs
as deep as 25 m in exposed environments or as shallow as 5 m in
protected environments (Rützler and Macintyre 1982, Aronson and Precht
2001, cited in \citet{Fredston_etal_2013}).

\emph{Acropora} and \emph{Porites} corals also dominated reefs in Bocas
del Toro during the late-Holocene (\textasciitilde{}4000-0 years ago,
xxxref) (Aronson et al. 2004, Cramer et al. 2012, cited in
\citet{Fredston_etal_2013}). Today, \emph{Porites} remains abundant and
\emph{Agaricia tenuifolia} (Dana, 1846) became important too (Guzman and
Guevara 1998) but \emph{Acropora} has catastrophically declined (Cramer
et al. 2012, cited in \citet{Fredston_etal_2013}).

If the ecological change observed in coral reefs from Bocas del Toro is
analysed in the context of the history of human occupation and impact,
it seems likely that water clarity had an importat role. Intensive land
clearing resulted in water to become more turbid, so \emph{Acropora}
corals (Rogers 1990, \citet{Cramer_etal_2012}), which love light were
replaced by \emph{Porites} and \emph{Agaricia} corals that resist
turbid-water better \citep{Cramer_etal_2012}.

\section{Resilience}\label{resilience}

Once they degrade severely, reefs can struggle to recover and,
eventually, reefs loose important ecosystem functions. Reefs from Bocas
del Toro were experimentally degraded and two years after their fate was
unclear: some recovered to pre-disturbance conditions but some did not
\citep{Schloder_etal_2013}. The impact on ecosystem functions may not be
immediate. Dead dead coral rubble can still host sessile infauna but
eventually the three-dimensional structure of the reef flattens and
looses biodiversity \citep{Nelson_etal_2016}. Reefs with high
three-dimensional structure are more resilient to bleaching (ref Graham
2015?).

In Bocas del Toro and many reefs worldwide, the greater builder of
three-dimensional has historically been\_Acropora\_ corals (xxx ref).
Because \emph{Acropora} is almost extinct in the Caribbean (xxx ref
Cramer?), the resilience of coral reefs here and in Bocas del Toro in
particular is seriously compromised.

\bibliography{packages,references}


\end{document}
